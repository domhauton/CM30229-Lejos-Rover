\documentclass[a4paper,10pt]{report}
\usepackage[utf8]{inputenc}

% Title Page
\title{CM30229 - Room Circumnavigation using LEJOS}
\author{Dominic Hauton}


\begin{document}
\maketitle

\begin{abstract}
\end{abstract}

\section{Introduction}
% The Introduction should give a brief description of what you have done, and also give some idea
% about why you have done it (motivation). I expect you to cite a paper or two for the research context.
% For coursework 1, one of the papers you should probably cite is Brooks (1991), since you have been
% asked to take a fairly reactive approach to developing robot intelligence.

\section{Approach}

% The approach describes in detail exactly what you have done. This section is longer, and should ideally
% include some experiments you set up, for example to determine in what conditions you could get better
% results from the robot. The approach should be in sufficient detail that another person could replicate
% your experiments. You may cite other papers here too if you are taking an approach from another
% paper, or modifying it only slightly.

% Courswork one is to construct a robot capable of circumnavigating rooms or other closed spaces
% (don’t worry about doorways – just close or block them.) Ideally this should work in “natural” (unaltered)
% indoor environments with a variety of obstacles along the walls. To quantify the outcomes of
% this coursework, you may want to think about questions such as contrasting the addition of extra control
% algorithms vs. changing the physical shape of the robot for increasing circuit time for the robot, or
% trying different target sonar readings for maintaining a particular distance from the wall in a variety of
% contexts. For coursework one it is quite likely that you will not have initially thought of a hypothesis
% to test, but will rather just have tried to make the robot work. However, in your exploration (both
% with the robot and with the reading) if you do find something that seems to make a difference, you
% should go try to capture what that something is. Can you describe it exactly? Can you replicate it with
% different robot configurations? Can you quantify how much improvement you get given how much
% change to some parameter on the robot? Don’t forget to consider things such as the state of battery
% charge or whether you are operating in daylight or in proximity to other sonar-using robots as possible
% explanations for strange behaviour.

% Please do mention who shared your robot in the approach section, and the extent to which you
% worked together. The objective here is to learn. How much you work together is totally up to you so
% long as you each write your report independently.

\section{Results}

% The results section describes the outcomes. This should be purely factual descriptions, including
% qualitative outcomes, quantitative outcomes and possibly statistics. For example, you could report the
% average speed around a circuit in two conditions plus standard deviations and a significance test to tell
% whether you have evidence that the conditions lead to different results. For coursework 1, this must
% include video. Typically, the results section can be surprisingly short, since the Approach section is
% the one giving details. Results are purely and only factual outcomes.

% With respect to your personal results, if you describe a reasonably-well working system in a comprehensible
% manner you will pass. If you competently fill in all of these sections as described in this
% specification, you will get at least 55. Getting a mark over 70 requires demonstrating insight, creativity
% and / or understanding that goes beyond the basics laid out for you in this document. For example, an
% insightful comment about one or more cited papers supported by evidence from your experience might
% get you these extra marks. So might a particularly accurate and replicable account of your approach
% and results.

\section{Discussion}

% The discussion is the most discursive part of your paper, it may include speculation. You should
% discuss the extent to which your results addressed the questions described in your introduction, and
% what the results imply about your own work and AI or robotics more broadly. You might suggest other
% experimental protocols that could have given different results and lessons learned. This can be a longer
% section as well.

\section{Conclusion}

% The conclusion is just one paragraph. After possible digressions in the discussion, you should come
% back to state exactly what you tried to do (brief summary of the introduction), what the outcome was
% (brief summary of the results), and what you can certainly state as a result of this (the implications of
% the results in light of the introduction.)

\end{document}          
